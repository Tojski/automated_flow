\chapter{Introduction}

\section{P3HT Synthesis}
Poly(thiophene)s are a class of organic polymer that exhibit semiconducting properties due to their delocalised -system. The polymer consists of repeating thiophene units, a sulphur –substituted pentene aromatic ring. Figure X shows the structure of the thiophene molecule, as well as polythiophene. 

\subsection{Polythiophene Synthesis}

The use of organic semiconductors to fabricate electronic devices has been widely popular since the first organic device in X [REF], with applications to solar cells [REF], LEDs [REF] and transistors [REF]. A variety of substituted polythiophenes have been applied to organic electronics, with papers drawing comparisons of different thiophene chemistries[REFs]. A polymer that displays attractive physical, chemical and electronic properties is poly-3-hexylthiophene, or P3HT[REF], which was the focus of the work described in this report. 
