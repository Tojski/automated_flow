\chapter{Introduction}

\section{P3HT Synthesis}
Poly(thiophene)s are a class of organic polymer that exhibit semiconducting properties due to their delocalised -system. The polymer consists of repeating thiophene units, a sulphur –substituted pentene aromatic ring. Figure X shows the structure of the thiophene molecule, as well as polythiophene. 

\subsection{Polythiophene}

The use of organic semiconductors to fabricate electronic devices has been widely popular since the first organic device in X [REF], with applications to solar cells [REF], LEDs [REF] and transistors [REF]. A variety of substituted polythiophenes have been applied to organic electronics, with papers drawing comparisons of different thiophene chemistries[REFs]. A polymer that displays attractive physical, chemical and electronic properties is poly-3-hexylthiophene, or P3HT[REF], which was the focus of the work described in this report. 

\subsection{PT Synthesis}

For the application of poly(thiophene) to organic electronics, the semiconducting material needs to be amenable to thin-film processing[REF]. The earliest synthesis of poly(thiophene) investigated by Yamamoto et. al. and Lin et. al. describes the relative insolubility of higher-weight PT in organic solvents. This limits the ability for solution processing, and therefore the ability to create thin films to test the conductivity of the material. The attention of poly(thiophenes) was then turned to alkyl-substituted thiophenes, as alkyl groups could increase the solubility of the polymer in common solvents such as THF to allow for solution processing.
C. Arbizzani et. al. published a comparison of the poly(alkylthiophenes) (PATs) poly(3-methyl-hexylthiophene) and P3HT. In the paper, he discussed the “increasing attention” toward longer chain PATs as a result of their better solution processability. (SOMETHING ABOUT THE WORSE ELECTRONIC PROPERTIES WITH TOO LONG ALKYL CHAINS)

\subsection{PAT Synthesis}

Early syntheses of PATs showed a wide range of chemistries. In an investigation in processing thin films of PT, Osawa et. al. demonstrated an electrochemical method of PAT synthesis, in which an ultrasonic field was placed over the reaction to afford a PAT film. An oxidative polymerisation utilising a FeCl3 catalyst was first published by Sugimoto et. al., which Leclerc et. al. demonstrated to exhibit a higher level of crystallinity than electrochemical methods. A third method, first published by Jen et. al., employed a nickel complex to catalyse the polymerisation of a thiophenyl Grignard reagent. This method, known as the Kumada cross-coupling method, can be seen in figure X.

The Kumada cross-coupling method can produce PATs of higher molecular weight than the FeCl3 or electrochemical methods, but due to the lack of selectivity of the catalyst, a mixture of stereochemistries is obtained. The different couplings of each additional monomer, shown in Figure X, result in a regiorandom polymer.

\subsection{Synthesis of Regioregular PAT}

Due to the asymmetric structure of a mono-substituted alkyl-thiophene, the addition of two monomers can result in three conformations, as demonstrated in the mechanism of the Kumada cross-coupling method. Conformation 2.2 is known as T-T coupling, where the two alkyl chains point away from one another. Conformation 2.3 is known as H-T coupling, where the alkyl chains lie in the same direction. Finally, conformation 2.4 is known as H-H coupling, where the alkyl chains point towards each other. H-H couplings introduce unfavourable interactions between the two alkyl chains because of steric hindrance. The result is a twisted polymer backbone, both breaking conjugation and reducing the crystallinity of the polymer. This therefore effects the conductivity of the PAT thin film, with a greater number of H-H couplings present in the polymer chain resulting in a lower conductivity. R. Maur et. al. published a report that demonstrates the effect of regioregularity in P3HT on solar cell efficiency indicating the importance of highly regioregular materials. 
Following on from the demand for regioregular PAT synthesis, a variety of synthetic methods were published, such as A. Iraqi et. al. who outlined a palladium catalysed pathway and S. Guillerez et. al. who followed a Suzuki coupling process. However, the two methods that provided the highest regioregularity were the Grignard metathesis with a nickel catalyst pioneered by McCullough et. al., and the organozinc metathesis published by Rieke et. al.

\subsection{McCullough Synthesis}

The synthesis outlined by McCullough et. al. can be seen in Figure X. A 5-bromo thiophene is reacted with the Grignard reagent MgBr2.OEt2 to generate the active monomer species. By performing the reaction at -78 C, the kinetic product X.2 forms more readily over the undesirable monomer X.3 in a 99:1 ratio. The nickel complex Ni(dppp)Cl2 is then used to polymerise the Grignard monomer via a cross-couple mechanism. The PATs synthesised by this method displayed a regioregularity (RR) of 98 \%.

\subsection{Rieke Synthesis}

The reaction scheme for the Rieke synthesis can be seen in Figure X. A dibromothiophene was reacted with “Rieke Zinc”, a fine powder of zinc to generate two organozinc species X.2 and X.3. The desired species, X.2, was formed in a 9:1 ratio to X.3. The polymerisation was then catalysed by Ni(dppe)Cl2 to afford PATs with a RR of >97 \%.

