\renewcommand{\abstractname}{Abstract}
\begin{abstract}
	\setcounter{page}{3}
	This project focussed on the development of a fully automated droplet flow reactor for the synthesis of semiconducting polymers. The system in question consists of a droplet flow reactor outlined by Bannock et. al., coupled to a prototype gel permeation chromatography instrument to determine the weight-average (Mw) and polydispersity (PD) of the synthesised product. Furthermore, a liquid-liquid droplet flow separator was designed that can monitor the degree of separation of the product material from the reaction carrier liquid to alert the user to improper separation of the product solution, as to not damage the GPC column. By providing a near-continual analysis of the product material, the flow rates of the reactants can be altered during the reaction to produce a polymeric product of desired Mw and PD.

The first part of the project involved providing a viable termination step for the GRIM polymerisation of P3HT in a flow synthesis to prevent gelling of the polymer in the GPC column during the analysis. Once this was achieved, the reaction was monitored through the GPC instrument for a long reaction time to demonstrate long-term usability of the column without blockage. 

The issue of imperfect separation was then addressed to highlight the user to unwanted carrier fluid entering the GPC column, causing a null reading. This was achieved through the development of a liquid-liquid droplet flow separator that monitors the phase of the two output streams by use of photodiodes and LEDs, and accordingly can control a valve to reach perfect separation. 

In the final part of the project, the prototype GPC instrument was calibrated against an Agilent GPC instrument to provide a measure of the product polymer’s properties throughout the synthesis, in preparation for integrating a self-optimising algorithm into the system.

\end{abstract}