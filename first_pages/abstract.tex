\renewcommand{\abstractname}{Abstract}
\begin{abstract}
	\setcounter{page}{3}
This project focussed on the development of an automated droplet flow reactor for the synthesis of semiconducting polymers. The system in question consists of a droplet flow reactor used to synthesise the semiconducting polymer poly(3-hexylthiophene) (P3HT) via a Grignard metathesis synthetic method outlined by Bannock et. al. This reactor is coupled to a prototype gel permeation chromatography (GPC) instrument to determine the number-average molecular weight (Mn) of the synthesised product. The aim of the work is to develop a flow reactor that can analyse its product in-line to make the system amenable to self-optimisation and remote control. 

A problem identified by Bannock et. al. with the droplet flow synthesis method was that upon leaving the reactor further polymerisation could occur in the reaction mixture, resulting in gelling of the polymer that could block the reactor and GPC column. The first part of the project involved providing a viable termination step for the Grignard Metathesis method of polymerisation of P3HT in a flow synthesis to prevent these blockages in the GPC column during the analysis. Once this was achieved, the reaction was monitored through the GPC instrument over a period of two hours to demonstrate long-term usability of the column without blockage. 

To allow for in-line analysis of the product of the droplet-flow reactor, a liquid-liquid separator is necessary following the reaction to obtain a pure stream of the reaction mixture, prior to analysis. The design supplied at the beginning of the project was not able to detect whether separation of the two-phase system occurred, which could interfere with or damage the analysis of the reaction mixture if left unattended. Therefore, a liquid-liquid separator was designed with the ability to monitor the degree of separation of the product material from the reaction carrier liquid to alert the user to improper separation of the product solution. This was achieved through the development of a liquid-liquid droplet flow separator that monitors the phase of the two output streams by use of photodiodes and LEDs. 

In the final part of the project, the prototype GPC instrument was calibrated against an Agilent GPC to allow for in-line determination of the product polymer’s Mn.


\end{abstract}